\chapter{Introduction}
\thispagestyle{empty}

\begin{section}{What is Abstract Algebra?}

Abstract algebra is the subject area of mathematics that studies algebraic structures, such as groups, rings, fields, modules, vector spaces, and algebras. This course is an introduction to abstract algebra. We will spend most of our time studying groups. Group theory is the study of symmetry, and is one of the most beautiful areas in all of mathematics. It arises in puzzles, visual arts, music, nature, the physical and life sciences, computer science, cryptography, and of course, throughout mathematics. This course will cover the basic concepts of group theory, and a special effort will be made to emphasize the intuition behind the concepts and motivate the subject matter.  In the last few weeks of the semester, we will also introduce rings and fields.

\end{section}

\begin{section}{An Inquiry-Based Approach}

In a typical course, math or otherwise, you sit and listen to a lecture. (Hopefully) These lectures are polished and well-delivered. You may have often been lured into believing that the instructor has opened up your head and is pouring knowledge into it. I absolutely love lecturing and I do believe there is value in it, but I also believe that in reality most students do not learn by simply listening. You must be active in the learning process. I'm sure each of you have said to yourselves, ``Hmmm, I understood this concept when the professor was going over it, but now that I am alone, I am lost." In order to promote a more active participation in your learning, we will incorporate ideas from an educational philosophy called inquiry-based learning (IBL).

Loosely speaking, IBL is a student-centered method of teaching mathematics that engages students in sense-making activities.  Students are given tasks requiring them to solve problems, conjecture, experiment, explore, create, communicate.  Rather than showing facts or a clear, smooth path to a solution, the instructor guides and mentors students via well-crafted problems through an adventure in mathematical discovery.  Effective IBL courses encourage deep engagement in rich mathematical activities and provide opportunities to collaborate with peers (either through class presentations or group-oriented work).

Perhaps this is sufficiently vague, but I believe that there are two essential elements to IBL.  Students should as much as possible be responsible for:
\begin{enumerate}
\item Guiding the acquisition of knowledge, and
\item Validating the ideas presented.  That is, students should not be looking to the instructor as the sole authority.
\end{enumerate}
\noindent For additional information, check out my blog post, \href{http://maamathedmatters.blogspot.com/2013/05/what-heck-is-ibl.html}{What the Heck is IBL?}

Much of the course will be devoted to students proving theorems on the board and a significant portion of your grade will be determined by how much mathematics you produce. I use the word ``produce" because I believe that the best way to learn mathematics is by doing mathematics. Someone cannot master a musical instrument or a martial art by simply watching, and in a similar fashion, you cannot master mathematics by simply watching; you must do mathematics!

Furthermore, it is important to understand that proving theorems is difficult and takes time. You should not expect to complete a single proof in 10 minutes. Sometimes, you might have to stare at the statement for an hour before even understanding how to get started. 

In this course, everyone will be required to
\begin{itemize}
\item read and interact with course notes on your own;
\item write up quality proofs to assigned problems;
\item present proofs on the board to the rest of the class;
\item participate in discussions centered around a student's presented proof;
\item call upon your own prodigious mental faculties to respond in flexible, thoughtful, and creative ways to problems that may seem unfamiliar on first glance.
\end{itemize}
\noindent As the semester progresses, it should become clear to you what the expectations are. This will be new to many of you and there may be some growing pains associated with it.

Lastly, it is highly important to respect learning and to respect other people's ideas.  Whether you disagree or agree, please praise and encourage your fellow classmates.  Use ideas from others as a starting point rather than something to be judgmental about.  Judgement is not the same as being judgmental.  Helpfulness, encouragement, and compassion are highly valued.

\end{section}

\begin{section}{Rules of the Game}
You should \emph{not} look to resources outside the context of this course for help. That is, you should not be consulting the Internet, other texts, other faculty, or students outside of our course. On the other hand, you may use each other, the course notes, me, and your own intuition.  In this class, earnest failure outweighs counterfeit success; you need not feel pressure to hunt for solutions outside your own creative and intellectual reserves.  For more details, check out the Syllabus.

\end{section}

\begin{section}{Structure of the Notes}

As you read the notes, you will be required to digest the material in a meaningful way.  It is your responsibility to read and understand new definitions and their related concepts.  However, you will be supported in this sometimes difficult endeavor. In addition, you will be asked to complete exercises aimed at solidifying your understanding of the material.  Most importantly, you will be asked to make conjectures, produce counterexamples, and prove theorems.

Most items in the notes are labelled with a number.  The items labelled as \textbf{Definition} and \textbf{Example} are meant to be read and digested.  However, the items labelled as \textbf{Exercise}, \textbf{Question}, \textbf{Theorem}, \textbf{Corollary}, and \textbf{Problem} require action on your part.  In particular, items labelled as \textbf{Exercise} are typically computational in nature and are aimed at improving your understanding of a particular concept.  There are very few items in the notes labelled as \textbf{Question}, but in each case it should be obvious what is required of you.  Items with the \textbf{Theorem} and \textbf{Corollary} designation are mathematical facts and the intention is for you to produce a valid proof of the given statement.  The main difference between a \textbf{Theorem} and \textbf{Corollary} is that corollaries are typically statements that follow quickly from a previous theorem.  In general, you should expect corollaries to have very short proofs.  However, that doesn't mean that you can't produce a more lengthy yet valid proof of a corollary.  The items labelled as \textbf{Problem} are sort of a mixed bag.  In many circumstances, I ask you to provide a counterexample for a statement if it is false or to provide a proof if the statement is true.  Usually, I have left it to you to determine the truth value.  If the statement for a problem is true, one could relabel it as a theorem.

It is important to point out that there are very few examples in the notes.  This is intentional.  One of the goals of the items labelled as \textbf{Exercise} is for you to produce the examples.

Lastly, there are many situations where you will want to refer to an earlier definition or theorem/corollary/problem.  In this case, you should reference the statement by number.  For example, you might write something like, ``By Theorem 1.13, we see that\ldots."

\end{section}

\begin{section}{Some Minimal Guidance}
Especially in the opening sections, it won't be clear what facts from your prior experience in mathematics we are ``allowed" to use.  Unfortunately, addressing this issue is difficult and is something we will sort out along the way.  However, in general, here are some minimal and vague guidelines to keep in mind.  

First, there are times when we will need to do some basic algebraic manipulations.  You should feel free to do this whenever the need arises.  But you should show sufficient work along the way.  You do not need to write down justifications for basic algebraic manipulations (e.g., adding 1 to both sides of an equation, adding and subtracting the same amount on the same side of an equation, adding like terms, factoring, basic simplification, etc.).  

On the other hand, you do need to make explicit justification of the logical steps in a proof.  When necessary, you should cite a previous definition, theorem, etc. by number.

Unlike the experience many of you had writing proofs in geometry, our proofs will be written in complete sentences.  You should break sections of a proof into paragraphs and use proper grammar.  There are some pedantic conventions for doing this that I will point out along the way.  Initially, this will be an issue that most students will struggle with, but after a few weeks everyone will get the hang of it.

Ideally, you should rewrite the statements of theorems before you start the proof.  Moreover, for your sake and mine, you should label the statement with the appropriate number.  I will expect you to indicate where the proof begins by writing ``\emph{Proof.}" at the beginning.  Also, we will conclude our proofs with the standard ``proof box" (i.e., \(\square\) or \(\blacksquare\)), which is typically right-justified.

Lastly, every time you write a proof, you need to make sure that you are making your assumptions crystal clear.  Sometimes there will be some implicit assumptions that we can omit, but at least in the beginning, you should get in the habit of stating your assumptions up front.  Typically, these statements will start off ``Assume\ldots" or ``Let\ldots".  

This should get you started.  We will discuss more as the semester progresses.  Now, go have fun and kick some butt!

\end{section}
