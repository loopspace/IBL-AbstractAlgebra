\chapter{An Intuitive Approach to Groups}
\label{chapter:intuitive_approach_groups}
\thispagestyle{empty}

One of the major topics of this course is \textbf{groups}.  The area of mathematics that is concerned with groups is called \textbf{group theory}. Loosely speaking, group theory is the study of symmetry, and is one of the most beautiful areas in all of mathematics. It arises in puzzles, visual arts, music, nature, the physical and life sciences, computer science, cryptography, and of course, throughout mathematics.

Instead of starting with an abstract formal definition, we will begin our study of groups by developing some intuition about what groups actually are.  To get started, we will be exploring the game Spinpossible.   Spinpossible is a free game that is available for iOS and Android devices. Alternatively, you can just play the game in any modern web browser (\url{http://spinpossible.com}).  The game is played on a \(3\times 3\) board of scrambled tiles numbered 1 to 9, each of which may be right-side-up or up-side-down. The objective of the game is to return the board to the standard configuration where tiles are arranged in numerical order and right-side-up. This is accomplished by a sequence of ``spins", where a spin consists of rotating an \(m\times n\) subrectangles by 180\(^\circ\). The goal is to minimize the number of spins used.  The following figure depicts a scrambled board on the left and the solved board on the right.  The sequence of arrows is used to denote some sequence of spins.

\begin{center}
\begin{tabular}{c}\includegraphics[width=1.5in]{scramble1.PNG}\end{tabular}
{\large \(\xrightarrow{?} \cdots \xrightarrow{?}\)}
\begin{tabular}{c}\includegraphics[width=1.5in]{scramble4.PNG}\end{tabular}
\end{center}

\begin{example}\label{ex:spinpossible}
Let's play with an example.  Suppose we start with the following scrambled board.

\begin{center}
\begin{tikzpicture}[every node/.style={minimum size=.65cm}]
  \node [draw] (1) {\rotatebox{180}{\(\underline{2}\)}};
  \node [draw, right=0cm of 1] (2) {\rotatebox{180}{\(\underline{9}\)}};
  \node [draw, right=0cm of 2] (3) {\rotatebox{180}{\(\underline{1}\)}};
  \node [draw, below=0cm of 1] (4) {\(\underline{4}\)};
  \node [draw, right=0cm of 4] (5) {\rotatebox{180}{\(\underline{6}\)}};
  \node [draw, right=0cm of 5] (6) {\(\underline{5}\)};
  \node [draw, below=0cm of 4] (7) {\(\underline{7}\)};
  \node [draw, right=0cm of 7] (8) {\rotatebox{180}{\(\underline{3}\)}};
  \node [draw, right=0cm of 8] (9) {\(\underline{8}\)};
\end{tikzpicture}
\end{center}

\noindent The underlines on the numbers are meant to help us tell whether a tile is right-side-up or up-side-down.  Our goal is to use a sequence of spins to unscrambled the board.  Before we get started, let's agree on some conventions.  When we refer to tile \(n\), we mean the actual tile that is labeled by the number \(n\) regardless of its position and orientation on the board.  On the other hand, position \(n\) will refer to the position on the board that tile \(n\) is supposed to be in when the board has been unscrambled.  For example, in the board above, tile 1 is in position 3 and tile 7 happens to be in position 7.  

It turns out that there are multiple ways to unscrambled this board., but I have one particular sequence in mind.  First, let's spin the rectangle determined by the two rightmost columns.  Here's what we get.  I've shaded the subrectangle that we are spinning.

\begin{center}
\begin{tabular}{c}
\begin{tikzpicture}[every node/.style={minimum size=.65cm}]
  \node [draw] (1) {\rotatebox{180}{\(\underline{2}\)}};
  \node [draw, fill=blue!40, right=0cm of 1] (2) {\rotatebox{180}{\(\underline{9}\)}};
  \node [draw, fill=blue!40, right=0cm of 2] (3) {\rotatebox{180}{\(\underline{1}\)}};
  \node [draw, below=0cm of 1] (4) {\(\underline{4}\)};
  \node [draw, fill=blue!40, right=0cm of 4] (5) {\rotatebox{180}{\(\underline{6}\)}};
  \node [draw, fill=blue!40, right=0cm of 5] (6) {\(\underline{5}\)};
  \node [draw, below=0cm of 4] (7) {\(\underline{7}\)};
  \node [draw, fill=blue!40, right=0cm of 7] (8) {\rotatebox{180}{\(\underline{3}\)}};
  \node [draw, fill=blue!40, right=0cm of 8] (9) {\(\underline{8}\)};
\end{tikzpicture}
\end{tabular}
%
{\large \(\rightarrow\)}
%
\begin{tabular}{c}
\begin{tikzpicture}[every node/.style={minimum size=.65cm}]
  \node [draw] (1) {\rotatebox{180}{\(\underline{2}\)}};
  \node [draw, right=0cm of 1] (2) {\rotatebox{180}{\(\underline{8}\)}};
  \node [draw, right=0cm of 2] (3) {\(\underline{3}\)};
  \node [draw, below=0cm of 1] (4) {\(\underline{4}\)};
  \node [draw, right=0cm of 4] (5) {\rotatebox{180}{\(\underline{5}\)}};
  \node [draw, right=0cm of 5] (6) {\(\underline{6}\)};
  \node [draw, below=0cm of 4] (7) {\(\underline{7}\)};
  \node [draw, right=0cm of 7] (8) {\(\underline{1}\)};
  \node [draw, right=0cm of 8] (9) {\(\underline{9}\)};
\end{tikzpicture}
\end{tabular}
\end{center}

\noindent Okay, now let's spin the middle column.

\begin{center}
\begin{tabular}{c}
\begin{tikzpicture}[every node/.style={minimum size=.65cm}]
  \node [draw] (1) {\rotatebox{180}{\(\underline{2}\)}};
  \node [draw, fill=blue!40, right=0cm of 1] (2) {\rotatebox{180}{\(\underline{8}\)}};
  \node [draw, right=0cm of 2] (3) {\(\underline{3}\)};
  \node [draw, below=0cm of 1] (4) {\(\underline{4}\)};
  \node [draw, fill=blue!40, right=0cm of 4] (5) {\rotatebox{180}{\(\underline{5}\)}};
  \node [draw, right=0cm of 5] (6) {\(\underline{6}\)};
  \node [draw, below=0cm of 4] (7) {\(\underline{7}\)};
  \node [draw, fill=blue!40, right=0cm of 7] (8) {\(\underline{1}\)};
  \node [draw, right=0cm of 8] (9) {\(\underline{9}\)};
\end{tikzpicture}
\end{tabular}
%
{\large \(\rightarrow\)}
%
\begin{tabular}{c}
\begin{tikzpicture}[every node/.style={minimum size=.65cm}]
  \node [draw] (1) {\rotatebox{180}{\(\underline{2}\)}};
  \node [draw, right=0cm of 1] (2) {\rotatebox{180}{\(\underline{1}\)}};
  \node [draw, right=0cm of 2] (3) {\(\underline{3}\)};
  \node [draw, below=0cm of 1] (4) {\(\underline{4}\)};
  \node [draw, right=0cm of 4] (5) {\(\underline{5}\)};
  \node [draw, right=0cm of 5] (6) {\(\underline{6}\)};
  \node [draw, below=0cm of 4] (7) {\(\underline{7}\)};
  \node [draw, right=0cm of 7] (8) {\(\underline{8}\)};
  \node [draw, right=0cm of 8] (9) {\(\underline{9}\)};
\end{tikzpicture}
\end{tabular}
\end{center}

\noindent Hopefully, you can see that we are really close to unscrambling the board.  All we need to do is spin the rectangle determined by the tiles in positions 1 and 2.

\begin{center}
\begin{tabular}{c}
\begin{tikzpicture}[every node/.style={minimum size=.65cm}]
  \node [draw, fill=blue!40] (1) {\rotatebox{180}{\(\underline{2}\)}};
  \node [draw, fill=blue!40, right=0cm of 1] (2) {\rotatebox{180}{\(\underline{1}\)}};
  \node [draw, right=0cm of 2] (3) {\(\underline{3}\)};
  \node [draw, below=0cm of 1] (4) {\(\underline{4}\)};
  \node [draw, right=0cm of 4] (5) {\(\underline{5}\)};
  \node [draw, right=0cm of 5] (6) {\(\underline{6}\)};
  \node [draw, below=0cm of 4] (7) {\(\underline{7}\)};
  \node [draw, right=0cm of 7] (8) {\(\underline{8}\)};
  \node [draw, right=0cm of 8] (9) {\(\underline{9}\)};
\end{tikzpicture}
\end{tabular}
%
{\large \(\rightarrow\)}
%
\begin{tabular}{c}
\begin{tikzpicture}[every node/.style={minimum size=.65cm}]
  \node [draw] (1) {\(\underline{1}\)};
  \node [draw, right=0cm of 1] (2) {\(\underline{2}\)};
  \node [draw, right=0cm of 2] (3) {\(\underline{3}\)};
  \node [draw, below=0cm of 1] (4) {\(\underline{4}\)};
  \node [draw, right=0cm of 4] (5) {\(\underline{5}\)};
  \node [draw, right=0cm of 5] (6) {\(\underline{6}\)};
  \node [draw, below=0cm of 4] (7) {\(\underline{7}\)};
  \node [draw, right=0cm of 7] (8) {\(\underline{8}\)};
  \node [draw, right=0cm of 8] (9) {\(\underline{9}\)};
\end{tikzpicture}
\end{tabular}
\end{center}

\noindent Putting all of our moves together, here is what we have.

\begin{center}
\begin{tabular}{c}
\begin{tikzpicture}[every node/.style={minimum size=.65cm}]
  \node [draw] (1) {\rotatebox{180}{\(\underline{2}\)}};
  \node [draw, fill=blue!40, right=0cm of 1] (2) {\rotatebox{180}{\(\underline{9}\)}};
  \node [draw, fill=blue!40, right=0cm of 2] (3) {\rotatebox{180}{\(\underline{1}\)}};
  \node [draw, below=0cm of 1] (4) {\(\underline{4}\)};
  \node [draw, fill=blue!40, right=0cm of 4] (5) {\rotatebox{180}{\(\underline{6}\)}};
  \node [draw, fill=blue!40, right=0cm of 5] (6) {\(\underline{5}\)};
  \node [draw, below=0cm of 4] (7) {\(\underline{7}\)};
  \node [draw, fill=blue!40, right=0cm of 7] (8) {\rotatebox{180}{\(\underline{3}\)}};
  \node [draw, fill=blue!40, right=0cm of 8] (9) {\(\underline{8}\)};
\end{tikzpicture}
\end{tabular}
%
{\large \(\rightarrow\)}
%
\begin{tabular}{c}
\begin{tikzpicture}[every node/.style={minimum size=.65cm}]
  \node [draw] (1) {\rotatebox{180}{\(\underline{2}\)}};
  \node [draw, fill=blue!40, right=0cm of 1] (2) {\rotatebox{180}{\(\underline{8}\)}};
  \node [draw, right=0cm of 2] (3) {\(\underline{3}\)};
  \node [draw, below=0cm of 1] (4) {\(\underline{4}\)};
  \node [draw, fill=blue!40, right=0cm of 4] (5) {\rotatebox{180}{\(\underline{5}\)}};
  \node [draw, right=0cm of 5] (6) {\(\underline{6}\)};
  \node [draw, below=0cm of 4] (7) {\(\underline{7}\)};
  \node [draw, fill=blue!40, right=0cm of 7] (8) {\(\underline{1}\)};
  \node [draw, right=0cm of 8] (9) {\(\underline{9}\)};
\end{tikzpicture}
\end{tabular}
%
{\large \(\rightarrow\)}
%
\begin{tabular}{c}
\begin{tikzpicture}[every node/.style={minimum size=.65cm}]
  \node [draw, fill=blue!40] (1) {\rotatebox{180}{\(\underline{2}\)}};
  \node [draw, fill=blue!40, right=0cm of 1] (2) {\rotatebox{180}{\(\underline{1}\)}};
  \node [draw, right=0cm of 2] (3) {\(\underline{3}\)};
  \node [draw, below=0cm of 1] (4) {\(\underline{4}\)};
  \node [draw, right=0cm of 4] (5) {\(\underline{5}\)};
  \node [draw, right=0cm of 5] (6) {\(\underline{6}\)};
  \node [draw, below=0cm of 4] (7) {\(\underline{7}\)};
  \node [draw, right=0cm of 7] (8) {\(\underline{8}\)};
  \node [draw, right=0cm of 8] (9) {\(\underline{9}\)};
\end{tikzpicture}
\end{tabular}
%
{\large \(\rightarrow\)}
%
\begin{tabular}{c}
\begin{tikzpicture}[every node/.style={minimum size=.65cm}]
  \node [draw] (1) {\(\underline{1}\)};
  \node [draw, right=0cm of 1] (2) {\(\underline{2}\)};
  \node [draw, right=0cm of 2] (3) {\(\underline{3}\)};
  \node [draw, below=0cm of 1] (4) {\(\underline{4}\)};
  \node [draw, right=0cm of 4] (5) {\(\underline{5}\)};
  \node [draw, right=0cm of 5] (6) {\(\underline{6}\)};
  \node [draw, below=0cm of 4] (7) {\(\underline{7}\)};
  \node [draw, right=0cm of 7] (8) {\(\underline{8}\)};
  \node [draw, right=0cm of 8] (9) {\(\underline{9}\)};
\end{tikzpicture}
\end{tabular}
\end{center}

\noindent In this case, we were able to solve the scrambled board in 3 moves.  It's not immediately obvious, but it turns out that there is no way to unscramble the board in fewer than 3 spins.  However, there is at least one other solution that involves exactly 3 spins.  We won't worry about proving this; right now we are just trying to gain some intuition.

\end{example}

\begin{exercise}
Without worrying about whether your solution is optimal, try to find a different sequence of spins that unscrambles the initial board in Example~\ref{ex:spinpossible}.  Your answer should be a sequence of spins.  Describe the spins in a way that makes sense.
\end{exercise}

\begin{exercise}\label{exer:number_spinpossible_boards}
How many scrambled boards are there?  To answer this question, you will need to rely on some counting principles such as factorials.
\end{exercise}

\begin{exercise}
A natural question to ask is whether every possible scrambling of a board in Spinpossible can be unscrambled using only spins.  It turns out that the answer is yes.  Justify this fact by describing an algorithm that will always unscramble a scrambled board.  It does not matter whether your algorithm is efficient.  That is, we don't care how many steps it takes to unscramble the board as long as it works in a finite number of steps.  Also, if it didn't occur to you yet, we can always spin a single tile (often referred to as \emph{toggling} a tile).
\end{exercise}

\begin{exercise}
Does the order in which you apply spins matter?  Does it always matter?  Let's be as specific as possible.  If the order in which we apply two spins does not matter, then we say that the tiles \textbf{commute}.  However, if the order does matter, then the spins do not commute.  When will two spins commute?  When will they not commute?  Provide some specific examples.
\end{exercise}

\begin{exercise}
How many possible spins are there?  We are referring to the moves you are allowed to do at any stage in the game.  Don't forget that you are allowed to toggle a single tile.
\end{exercise}

In a 2011 paper, Alex Sutherland and Andrew Sutherland (a father and son team) present a number of interesting results about Spinpossible and list a few open problems. You can find the paper at \url{http://arxiv.org/abs/1110.6645}. As a side note, Alex is one of the developers of the game and his father, Andrew, is a mathematics professor at MIT. Using brute-force and a computer algorithm, the Sutherlands verified that every scrambled board can be solved in at most 9 moves. However, a human readable mathematical proof of this fact remains elusive.  One of my undergraduate research students (Dane Jacobson) is currently working on this and some related open problems.  By the way, mathematics is chock full of open problems and you can often get the frontier of what is currently known without too much trouble.  Mathematicians are in the business of solving open problems.

At least for now, let's ignore the optimality requirement of the game.  That is, let's not worry about how many spins it takes to solve a scrambled board.  It turns out that we can ``build" some spins from other spins.  As an example, if I wanted to toggle the tile in position 2, I could first spin the rectangle determined by tiles 1 and 2, then toggle tile 1, and lastly spin the rectangle determined by tiles 1 and 2 again.  Of course, this is horribly inefficient, but it works.  Also, it is important to point out that I was describing the tile positions we were spinning while not paying any attention to the labels of the tiles.

It's not too difficult to prove that we can build all of the possible spins by only using the following spins.  I've listed some shorter names for these spins in parentheses.
\begin{enumerate}
\item Toggle position 1 (\(t\)),
\item Spin rectangle determined by positions 1 and 2 (\(s_1\)),
\item Spin rectangle determined by positions 2 and 3 (\(s_2\)),
\item Spin rectangle determined by positions 3 and 6 (\(s_3\)),
\item Spin rectangle determined by positions 6 and 5 (\(s_4\)),
\item Spin rectangle determined by positions 5 and 4 (\(s_5\)),
\item Spin rectangle determined by positions 4 and 7 (\(s_6\)),
\item Spin rectangle determined by positions 7 and 8 (\(s_7\)),
\item Spin rectangle determined by positions 8 and 9 (\(s_8\)).
\end{enumerate}
We can describe any of the allowable spins in the game by writing down a sequence consisting of \(t,s_1,\ldots, s_8\).  In fact, and more importantly, we can desribe any possible rearrangement of tiles (position and/or orientation) using just these 9 spins.  It is worth pointing out that not every sequence of these 9 spins results in an allowable spin.  We say that the set \(\{t, s_1,\ldots,s_8\}\) \textbf{generates} all possible scramblings of the \(9\times 9\) board.  It turns out that this generating set is minimal in the sense that we if tried to get rid of any one of \(t, s_1, \ldots, s_8\), we would no longer be able to generate all scramblings.  Note that there are other minimal generating sets and there are lots of sets that will generate all the scramblings that aren't minimal.

We need to establish some conventions about how to write down sequences of spins involving the generators.  Since we are doing spins on top of spins, we will follow the convention of function notation that says the function on the right goes first.  For example, \(ts_1 s_3\) means do \(s_3\) first, then do \(s_1\), and lastly do \(t\).  This will take some getting used to, but just remember that it is just like function notation (stuff on the right goes first).  We will refer to sequences like \(ts_1 s_3\) as \textbf{words} in the generators \(t,s_1,\ldots, s_8\).

\begin{exercise}
Try to write the spin that toggles position 3 as a sequence of moves involving only \(t, s_1, \ldots, s_8\).
\end{exercise}

\begin{exercise}
Try to write the spin that rotates the top row (i.e., spin the top row) as a sequence of moves involving only \(t, s_1, \ldots, s_8\).
\end{exercise}

Let's make a couple more observations.  First, every spin is reversible.  In this case, we could just apply the same spin again to undo it.  For example, \(s_1s_1\) is the same as doing nothing.\footnote{\emph{Warning:} Remember that we are exploring the game Spinpossible.  It won't always be the case that repeating a generator will reverse the action.}

\begin{exercise}
Imagine we started with a scrambled board and you were then able to unscramble the board using only \(t, s_1, \ldots, s_8\).  In this case, you would have some word in \(t, s_1, \ldots, s_8\). Let's call it \(w\).  Now, imagine you have the solved board.  How could you obtain the scrambled board that \(w\) unscrambled using only \(t, s_1,\ldots, s_8\)?
\end{exercise}

The upshot of the previous exercise is that the action of any sequence of generators can be reversed.

At this time, I think we are ready to summarize some of our observations of the game Spinpossible and to make a few general claims, which we will state as a list of rules.

\begin{description}
\item[Rule 1.] There is a predefined list of actions that never changes.
\item[Rule 2.] Every action is reversible.
\item[Rule 3.] Every action is deterministic.\footnote{By deterministic, we mean that we know exactly what will happen when we we apply an action.  In contrast, pulling a card of the top of shuffled deck of cards is not deterministic because we don't know which card we will end up with.}
\item[Rule 4.] Any sequence of consecutive actions is also an action.
\end{description}

Alright, here is our intuitive and unofficial definition of a group.

\begin{intuitivedef}\label{def:informal_group}
A \textbf{group} is a system or collection of actions that satisfies Rules 1--4 above.
\end{intuitivedef}

\begin{exercise}
Come up with a few examples that satisfy Rules 1--4 and a few examples that fail at least one of the rules.
\end{exercise}

\begin{exercise}\label{exer:2coins}%Exercise 1.1 in 1.5 of VGT
Place a penny and a nickel side by side on a table.  Consider just one action: swapping the positions of the two coins.  Is this a group?  Explain your answer.
\end{exercise}

\begin{exercise}
Consider Exercise~\ref{exer:2coins}, but add a dime to the right of the other two coins.  The only action is still the one from the previous exercise.  Is this a group?   Explain your answer.
\end{exercise}

\begin{exercise}
Consider your three coins from the previous exercise.  Now, for your actions take all possible rearrangements of the coins.  It turns out that this is a group.
\begin{enumerate}
\item[(a)] One of the actions is to swap the second and third coins.  What happens if you do this action twice?  Is this an action?  
\item[(b)] How many actions does this group have?  Describe them all.
\item[(c)] Can you think of a small set of actions that would generate all the other actions?  Can you find a minimal one?  Write all the actions of this group as a word in your generators?  Do some actions have more than one word representing it?
\end{enumerate}
\end{exercise}

\begin{exercise}
Will the sort of thing that happened in part (a) of the previous exercise happen in every group?
\end{exercise}

\begin{exercise}%Exercise 1.3 in in 1.5 of VGT
Imagine you have 10 coins in your left pocket.  Consider two actions: (1) move a coin from your left pocket to you right pocket, and (2) move a coin from your right pocket to your left pocket.  Is this a group?  Explain your answer.
\end{exercise}

\begin{exercise}
Imagine you have a square puzzle piece that fits perfectly in a square hole.  Consider these actions: pick up the square and rotate it an appropriate amount so that it fits back in the hole.  Is this a group?  Explain your answer.  If it is a group, how many distinct actions are there?
\end{exercise}

\begin{exercise}
Can you describe a group that has exactly \(n\) actions for any natural number \(n\)?
\end{exercise}

\begin{exercise}
Can you describe a situation that satisfies Rules 1--3, but not Rule 4.
\end{exercise}

\begin{exercise}\label{exer:introducing_Z}%Exercise 1.14 in in 1.5 of VGT
Pick your favorite integer.  Consider these actions: add any integer to the one you chose.  This is an infinite set of actions.  Is this a group?  If so, how small a set of generators can you find?
\end{exercise}

\begin{exercise}
Consider the previous exercise, but this time multiply instead of add.  Is this a group?  Explain your answer.
\end{exercise}
